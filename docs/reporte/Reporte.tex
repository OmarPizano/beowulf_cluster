\documentclass[journal]{IEEEtran}
\usepackage[utf8]{inputenc}
\usepackage{indentfirst}
\usepackage{graphicx}
\usepackage{float}
\usepackage{listings}
\usepackage{color}
\usepackage{csquotes}
\usepackage{url}
\renewcommand{\arraystretch}{1.5}	% padding de las tablas
\renewcommand{\figurename}{Figura}
\renewcommand{\lstlistingname}{Código}
\renewcommand{\abstractname}{Resumen}
\renewcommand{\IEEEkeywordsname}{Palabras Clave}
\renewcommand{\tablename}{Tabla}
\renewcommand{\refname}{Referencias}
\usepackage[font={small,it}]{caption}
\definecolor{light-gray}{gray}{0.95}
\lstset{
	basicstyle=\scriptsize\ttfamily,
	numbers=left,
	numberstyle=\scriptsize\ttfamily,
	numbersep=5pt,
	breaklines=true,
	backgroundcolor=\color{light-gray},
	xleftmargin=1cm,
	xrightmargin=1cm
%}
}

\begin{document}

\title{Diseño de un Clúster Beowulf GNU/Linux para el Modelado de Estructuras Químicas}

\author{
    \IEEEauthorblockN{
        O. J. Pizano\IEEEauthorrefmark{1},
        D. A. Gómez Martínez\IEEEauthorrefmark{2},
        E. Moreno Martínez\IEEEauthorrefmark{3},
        O. Ortiz Jimenez\IEEEauthorrefmark{4} y
        M. Trejo Durán\IEEEauthorrefmark{5}
    }
    \IEEEauthorblockA{
        Departamento de Estudios Multidisciplinarios,
        Universidad de Guanajuato\\
        Yuriria, Gto.\\
        \IEEEauthorrefmark{1}oj.pizanoorozco@ugto.mx,
        \IEEEauthorrefmark{2}da.gomezmartinez@ugto.mx,
        \IEEEauthorrefmark{3}e.morenomartinez@ugto.mx,
        \IEEEauthorrefmark{4}o.ortiz@ugto.mx,
        \IEEEauthorrefmark{5}mtrejo@ugto.mx
    }
}

\markboth{SEPTIEMBRE 2022}{} % no quitar estos los {}

\maketitle

\begin{abstract}
    Aquí vamos a hablar de los resultados que obtuvimos en las pruebas y los sistemas y servicios utilizados.
\end{abstract}

\begin{IEEEkeywords}
    Cluster, Beowulf, Diskless, Parallel Programming, Debian, GNU/Linux, PXE, DHCP, TFTP, SSH, NFS, OpenMPI, Quantum Espresso, Gaussian 09.
\end{IEEEkeywords}

\section{Introducción}
Dar contexto
\\
Hablar sobre las estructuras químicas?
\\
Qué es la arquitectura beowulf
\\
Arquitectura beowulf como solución barata
\\
Que es Linux y por qué usarlo

\section{Antecedentes}

Investigar y mencionar a grandes razgos trabajos ya realizados.

\section{Marco Teórico}

Justificación del problema (cuál es?) y la solución propuesta a grandes razgos.
\\
Conceptos fundamentales (beowulf, parallel prog, message passing interfacem, etc.)
\\
Hablar también sobre estructuras químicas?

\section{Diseño}

Describir los puntos críticos de la configuración, cosas no tan técnicas y más generales como la diferencia entre el masternode y el slave, el uso de un directorio compartido (CLOUD) para el MPI, etc.
\\
Poner un enlace al repositorio donde se puedan encontrar detalles técnicos específicos sobre la configuración de las siguientes tecnologías.

\subsection{Equipo}

Detalles sobre el equipo utilizado (CPU, memoria, etc.)

\subsection{Topología}

Describir (imagen) la topología de red del cluster y como ésta encaja en una red de propósito general.

\subsection{Sistema Operativo}

Hablar sobre el sistema operativo utilizado y puntos críticos sobre la configuración.

\subsection{Servicios}

Describir a grandes razgos los servicios (PXE, DHCP, NFS, etc.), por qué son necesarios y cuáles son los puntos importantes sobre esa configuración.

\subsection{Librería MPI}

Hablar sobre OpenMPI, sobre cómo se instala y como funciona de manera general.

\subsection{Software de Modelado de Estructuras Químicas}

Los diferentes softwares de modelado que van a utilizarse y mencionar los puntos críticos sobre su instalación y/o configuración.

\section{Escenario de Pruebas}

Describir a grandes razgos la metodología para realizar las pruebas con el software de modelado y la captura de resultados. Así como el software adicional utilizado para el monitoreo del cluster.

\section{Resultados}

Gráficos comparativos del rendimiento (tiempo CPU, memoria, etc.) en varias instancias (si aplica) con un CPU normal (p. ej. masternode), contra el cluster a su maxima potencia, etc.

\section{Conclusiones}

Bla bla y recomendaciones.

\begin{thebibliography}{10}
	\bibitem{article}
	J. Weizenbaum,
	"ELIZA—a computer program for the study of natural language communication between man and machine",
	\textit{Communications of the ACM},
	vol. 9,
	no. 1,
	pp. 36-45,
	1966.
	\bibitem{book}
	B. Ward,
	\textit{How Linux works: what every superuser should know}.
	3ra ed.
	CA: No Starch Press,
	2021.
	pp 123-141.
	\bibitem{website}
	Malathi T,
	"Setup a blog in minutes with Jekyll \& Github",
	2020.
	[En Línea].
	Disponible en: \url{https://www.loginradius.com/blog/async/setup-blog-in-minutes-with-jekyll/}.
	[Accedido: 13-Ene-2021]
\end{thebibliography}

\end{document}

%%%%%%%%%%%%%%%%%%%%%%%%%%%%%%%%%%%%%% EJEMPLOS

% EJEMPLO DE IMAGEN
%\begin{figure}[H]
%	\centering
%	\includegraphics[scale=0.6]{figures/model-pca-results.png}
%	\caption{Texto de la imagen}
%   \label{fig:figura}
%\end{figure}

% EJEMPLO DE TABLA
%\begin{table}[h]
%	\begin{tabular}{|c|c|c|c|p{5cm}|}
%		\hline
%		\textbf{Stakeholder} & \textbf{Rol} & \textbf{Interés} & \textbf{Influencia} & \textbf{Descripción} \\
%		\hline
%		Persona1 & Supervisor de Proyecto & Alto & Alto & Supervisa el trabajo en equipo de los desarrolladores. \\
%		\hline
%		Persona2 & Programador & Alto & Alto & Investiga el desarrollo del proyecto. \\
%		\hline
%		Persona3 & Programador & Alto & Alto & Investiga el desarrollo del proyecto. \\
%		\hline
%		Persona4 & Programador & Alto & Alto & Investiga el desarrollo del proyecto. \\
%		\hline
%	\end{tabular}
%	\caption{Stakeholders del proyecto.}
%	\label{tab:stakeholders}
%\end{table}

%%% EJEMPLO DE ECUACIONES
%\begin{center}
%    \begin{equation}
%        \vec{R}(t)=2e^t\sin(t)\hat{i}+3e^t\cos(t)\hat{j}+4e^t\hat{k}
%    \end{equation}
%    \begin{equation}
%        \vec{R}(t)=\sqrt{t^2-5}\hat{i}+\ln(t-8)\hat{j}+(t^2-3t+7)\hat{k}
%    \end{equation}
%\end{center}

%%% EJEMPLO DE CÓDIGO
%\begin{lstlisting}[caption=Ejemplo de Código en C.]
%#include <stdio.h>
%int main(int argc, const char *argv[])
%{
%	printf("Hello World!\n");
%	return 0;
%}
%\end{lstlisting}

%%% EJEMPLO DE CÓDIGO DESDE ARCHIVO EXTERNO
%\begin{lstlisting}[caption=Ejemplo de Código en C.]{code.c}
